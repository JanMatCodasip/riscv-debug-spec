\chapter{Future Ideas}
\label{sec:future}

\textbf{All items in this section are future ideas and should not be considered part of the specification.}

Some future version of this spec may implement some of the following features.

\begin{enumerate}
   \item The spec defines several additions to the Device Tree which enable a
      debugger to discover hart IDs and supported triggers for all the harts
      in the system.
   \item DTMs can function as general bus subordinate devices, so they would look like
      regular RAM to bus coordinators.
   \item Harts can be divided into groups. All the harts in the same group can
      be halted/run/stepped simultaneously. When a hart hits a breakpoint, all
      the other harts in the same group also halt within a few clock cycles.
   \item DTMs are specified for protocols like USB, I2C, SPI, and SWD.
   \item The debugger can communicate with the power manager to power cores up
      or down, and to query their status.
   \item Serial ports can raise an interrupt when a send/receive queue becomes full/empty.
   \item The debug interrupt can be masked by running code. If the interrupt is
      asserted, then deasserted, and then asserted again the debug interrupt
      happens anyway. This mechanism can be used to e.g.\ read/write memory with
      minimal interruption, making sure never to interrupt during a critical
      piece of code.
   \item The Debug Module can include a serial interface for re-using
      the DTM interface as a generic communication interface.
\end{enumerate}

\section{Serial Ports}

The Debug Module may implement up to 8 serial ports. They support basic flow
control and full duplex data transfer between a component and the debugger,
essentially allowing the Debug Transport to be used to communicate
with a debug monitor running on a hart, or more generally emulate devices which
aren't present. All these uses require software support, and are not further specified here.
Only the DMI side of the Debug Module serial registers are defined in this
specification as the core side interface should look like a peripheral device.

\input{serial.tex}
